\documentclass{article}
\title{CSCD350, Holiday Mailer: Software Requirements Specification}
\date{\today}

\begin{document}
\section{Introduction}
%This section provides an overview of the entire requirement document. This document describes all data, functional and behavioral requirements for software.

\subsection{Goals and objectives}
To build a program that stores addresses of friends and relatives and allows the user to send a 'Holiday Email' to people of her/his choice.

\subsection{Statement of scope}
%A description of the software is presented. Major inputs, processing functionality and outputs are described without regard to implementation detail.

\subsection{Software context}
%The software is placed in a business or product line context. Strategic issues relevant to context are discussed. The intent is for the reader to understand the 'big picture'.
Consumer grade product that is intuitive for use by the average computer user. The software handles their 'Holiday Letters' for them electronically.

\subsection{Major constraints}
%Any business or product line constraints that will impact the manner in which the software is to be specified, designed, implemented or tested are noted here.
The software will need to be able to run minimally on a consumer grade machine so as to not noticeably impact performance.
Web connection while running to allow access to the external e-mail servers.

\section{Usage scenario}
%This section provides a usage scenario for the software. It organized information collected during requirements elicitation into use-cases.


\subsection{User profiles}
%The profiles of all user categories are described here.


\subsection{Use-cases}
%All use-cases for the software are presented
Use case List all stored entries
Actor user
Basic display of all entries stored by the program

Use case List entries that sent letter recently
Actor user
Basic display of all entries that recently sent the user a letter

Use case List entries sorted by first name
Actor user
Basic display of all entries sorted by first name

Use case List entries sorted by last name
Actor user
Basic display of all entries sorted by last name

Use case List entries sorted by received letter date
Actor user
Basic display of all entries sorted by date of last letter received

Use case Add new person to database
Actor user
Basic choose to enter new person, enter first name, enter last name, enter address, enter if and email was sent, submit to database

Use case Remove new person to database
Actor user
Basic choose to remove, choose who to remove, make sure, remove
 
Use case Display entry’s by last name
Actor user
Basic choose to display, choose to display by last name, display
 
Use case Display entry’s by first name
Actor user
Basic choose to display, choose to display by first name, display
 
Use case Display entry’s with last name with certain letter
Actor user
Basic choose to display, choose to display by certain letters, display
 
Use case send email to everyone
Actor user
Basic choose to send email to everyone, pick holiday email to send, send
 
Use case send email to everyone who has sent one
Actor user
Basic choose to send email to everyone who has sent you one, pick holiday email to send, send
 
Use case send email to specific people
Actor user
Basic choose to send email to specific people, pick people, pick holiday email to send, send


\subsection{Special usage considerations}
%Special requirements associated with the use of the software are presented.

\section{Data Model and Description}
%This section describes information domain for the software

\subsection{Data Description}
%Data objects that will be managed/manipulated by the software are described in this section.

\subsubsection{Data objects}
%Data objects and their major attributes are described.

\subsubsection{Relationships}
%Relationships among data objects are described using an ERD- like form. No attempt is made to provide detail at this stage.

\subsubsection{Complete data model}
%An ERD for the software is developed

\subsubsection{Data dictionary}
%A reference to the data dictionary is provided. The dictionary is maintained in electronic form.

\section{Functional Model and Description}
%A description of each major software function, along with data flow or class hierarchy (OO) is presented.

\subsection{Description for Function n}
%A detailed description of each software function is presented. Section 4.1 is repeated for each of n functions.

\subsubsection{Processing narrative (PSPEC) for function n}
%A processing narrative for function n is presented.

\subsubsection{Function n flow diagram}
%A diagram showing the flow of information through the function and the transformation it undergoes is presented.

\subsubsection{Function n interface description}
%A detailed description of the input and output interfaces for the function is presented.

\subsubsection{Function n transforms}
%A detailed description for each transform (subfunction) for function n is presented. Section 4.1.4 is repeated for each of k transforms.

%Transform k description (processing narrative, PSPEC)

%Transform k interface description

%Transform k lower level flow diagrams

%Transform k interface description

\subsubsection{Performance Issues}
%Special performance required for the subsystem is specified.

\subsubsection{Design Constraints}
%Any design constraints that will impact the subsystem are noted.

\subsection{Software Interface Description}
%The software interface(s)to the outside world is(are) described.

\subsubsection{External machine interfaces}
%Interfaces to other machines (computers or devices) are described.

\subsubsection{External system interfaces}
%Interfaces to other systems, products or networks are described.

\subsubsection{Human interface}
%An overview of any human interfaces to be designed for the software is presented.

\subsection{Control flow description}
%The control flow for the system is presented with reference to Section 5.0 of this document.

\section{ Behavioral Model and Description}
%A description of the behavior of the software is presented.

\subsection{Description for software behavior}
%A detailed description of major events and states is presented in this section.

\subsubsection{Events}
%A listing of events (control, items) that will cause behavioral change within the system is presented.

\subsubsection{States}
%A listing of states (modes of behavior) that will result as a consequence of events is presented.

\subsection{State Transition Diagrams}
%Depict the overall behavior of the system.

\subsection{Control specification (CSPEC)}
%Depict the manner in which control is managed by the software.

\section{Restrictions, Limitations, and Constraints}
%Special issues which impact the specification, design, or implementation of the software are noted here.

\section{Validation Criteria}
%The approach to software validation is described.

\subsection{Classes of tests}
%The types of tests to be conducted are specified, including as much detail as is possible at this stage. Emphasis here is on black- box testing.

\subsection{Expected software response}
%The expected results from testing are specified.

\subsection{Performance bounds}
%Special performance requirements are specified. 

\section{Appendices}
%Presents information that supplements the Requirements Specification

\subsection{System traceability matrix}
%A matrix that traces stated software requirements back to the system specification.

\subsection{Product Strategies}
%If the specification is developed for a product, a description of relevant product strategy is presented here.

\subsection{Analysis metrics to be used}
%A description of all analysis metrics to be used during the analysis activity is noted here.

\subsection{Supplementary information (as required)}

\end{document}